 % AUTHOR: Diego Sarceno
% Last Update: 11.07.2020

\documentclass[11pt, spanish, letterpage]{article} %tipo de documento

\usepackage[letterpaper]{geometry} %margenes
\geometry{verbose,tmargin=2.5cm,bmargin=2.5cm,lmargin=2cm,rmargin=2cm}
\usepackage{amsmath,amsthm,amssymb} %modos matemáticos y  simbolos
\usepackage{latexsym,amsfonts} %simbolos matematicos
\usepackage{cancel} %hacer la linea que cancela las ecuaciones
\usepackage[spanish, es-noshorthands]{babel} %comandos en español y cambia el cuadro por la tabla
\decimalpoint %cambia las comas por puntos decimal
\usepackage[utf8]{inputenc} %caracteristicas del español
\usepackage{physics} %Simbolos fisicos
\usepackage{array} %mejores formatos de tabla
\parindent =0cm %sangria
\usepackage{graphicx} %graficas e imagenes
\usepackage{mathtools}
\usepackage[framemethod=TikZ]{mdframed}%Entornos talegas
\usepackage[colorlinks = true,
			linkcolor = blue,
			citecolor = black,
			urlcolor = blue]{hyperref}%formato de los links y URL's
\usepackage{multicol} %varias columnas
\usepackage{enumerate} %enumeraciones
\usepackage{pgf,tikz,pgfplots} %documentos en formato tikz
\usepackage{mathrsfs} %letras chingonas (transformada de laplace)
\usepackage{subfigure} %varias figuras seguidas
\usepackage[square,numbers]{natbib} %bibliografias
\usepackage[nottoc]{tocbibind}
\bibliographystyle{plainnat}
\usetikzlibrary{arrows, babel, calc}
\usepackage{tabulary}
\usepackage{multirow} %ocupar varias filas en una tabla
\usepackage{fancybox} %recuadros talegas
\usepackage{float} %ubicar graficas
\usepackage{color}
\usepackage{comment}
\usepackage{stackrel}
\usepackage{calligra}
\usepackage{lipsum} % texto de relleno
\usepackage{cite}
\usepackage{circuitikz} % crear circuitos
\usepackage{listings} % permite el ingreso de codigo
\usepackage{longtable}
%\usepackage{showframe}
%\usepackage{LobsterTwo}
% NEW PACKAGES
\usepackage{makeidx}
\usepackage{authblk} % para la manipulación de autores y afiliación
\usepackage{booktabs}
\usepackage{colortbl}
\usepackage{bbold}
\usepackage{dsfont}
\usepackage{tensor}
\usepackage{colortbl}
\usepackage{amsbsy}
\usepackage[draft,inline,nomargin]{fixme} \fxsetup{theme=color}

%This defines my comments
\definecolor{mycolor}{RGB}{0,0,250}
\FXRegisterAuthor{ds}{sds}{\color{mycolor}DS}


\usepackage{pdfpages}
\setlength{\parindent}{1cm} %sangria

%%%%%%%%%%%%%%%%%%%%%%%%%%%%%%%%%%%%%%%%%%%%%%%%%%%%%%%%%%%
\lstset{basicstyle=\ttfamily,breaklines=true}
\lstset{numbers=left, numberstyle=\tiny, stepnumber=1, numbersep=6pt}
\lstset{emph={import,as,return,for,in,else,if,def,True,const,False,append}, emphstyle=\color{blue}, emph={[2]pKronecker},
emphstyle={[2]\color{violet}}, emph={[3]float,input,int,range,print,len,double},
emphstyle={[3]\color{violet}}}
\lstset{morecomment=[l][\color{gray!40}]{\#}, morestring=[b][\color{green!50!black}]"}
%%	Importe de archivo: \lstinputlisting[inputencoding=latin1]{'nombre del archivo'.py}
%%%%%%%%%%%%%%%%%%%%%%%%%%%%%%%%%%%%%%%%%%%%%%%%%%%%%%%%%%%
\setlength{\columnseprule}{0pt}
%-------------------------------------------------
%\newcommand{\N}{\mathbb{N}}
%\newcommand{\Z}{\mathbb{Z}}
%\newcommand{\Q}{\mathbb{Q}}
%\newcommand{\I}{\mathbb{I}}
%\newcommand{\R}{\mathbb{R}}
%\newcommand{\C}{\mathbb{C}} %Conjuntos numericos
%\newcommand{\F}{\mathbb{F}} %Campo Cualquiera
%\newcommand{\Pos}{\mathbb{P}} %Reales positivos

\def\do#1{\csdef{#1}{\mathbb{#1}}}
\docsvlist{N,Z,Q,I,R,C,F,P}


\newcommand{\Hilbert}{\mathcal{H}} % Espacio de Hilbert
\newcommand{\f}{\textit{f}} %f de funcion
\newcommand{\g}{\textit{g}}
\newcommand{\kernel}{\mathscr{N}} %kernel
\newcommand{\range}{\mathcal{R}} %range
\newcommand{\lagran}{\mathcal{L}} %lagrangiano
\newcommand{\laplace}{\mathscr{L}} %transformada de laplace, mapas lineales
\newcommand{\partition}{\mathfrak{z}} % función de partición
\newcommand{\M}{\mathcal{M}} %Matrices
\newcolumntype{E}{>{$}c<{$}} %entorno matematico en columnas de una tabla
\newcommand{\vi}{\boldsymbol{\hat{\imath}}}
\newcommand{\vj}{\boldsymbol{\hat{\jmath}}}
\newcommand{\vk}{\vu{k}}%vectores unitarios R3
\newcommand{\vr}{\hat{r}}
\newcommand{\vp}{\boldsymbol{\hat{\phi}}}
\newcommand{\vz}{\vu{z}}%vectores unitarios en cilindricas
\newcommand{\vaz}{\boldsymbol{\hat{\theta}}}%vectores unitarios en esféricas
\newcommand{\vx}{\vu{x}}%vectores
\newcommand{\vy}{\vu{y}}%vectores 
\newcommand\numberthis{\addtocounter{equation}{1}\tag{\theequation}}
\newcommand{\LI}{\lim _{h\longrightarrow 0}}
\newcommand{\SU}{\longrightarrow \sum _{n=0} ^{\infty}}
\newcommand{\QED}{\hfill {\qed}}
\newcommand{\cis}{\text{cis} \,}
% matrices de pauli
\newcommand{\pauli}[1]{\sigma _{#1}}
%----------------------------------------------------------
%----------------------------------------------------------


%-paquete para unidades en el sistema internacional
\usepackage[load=prefix, load=abbr, load=physical]{siunitx}
\newunit{\gram}{g }%gramos
\newunit{\velocity}{ \metre / \Sec }%unidades de velocidad sistema internacional
\newunit{\acceleration}{ \metre / \Sec^2 }%unidades de aceleracion sistema internacional
\newunit{\entropy}{ \joule / \kelvin }%unidades de entropia sistema internacional
%--definiendo constantes fisicas en el SI
\newcommand{\accgravity}{9.8 \metre / \Sec^2}
%---diferencial inexacta
\newcommand{\dbar}{\mathchar'26\mkern-12mu d}
%-------------------------END-------------------------------------
%------------------------Barra negra-------------------------------
\tikzset{
	warningsymbol/.style={
		rectangle,draw=black,
		fill=white,scale=1,
		overlay}}
\mdfdefinestyle{warning}{%
	hidealllines=true,leftline=true,
	skipabove=12,skipbelow=12pt,
	innertopmargin=0.4em,%
	innerbottommargin=0.4em,%
	innerrightmargin=0.7em,%
	rightmargin=0.7em,%
	innerleftmargin=1.7em,%
	leftmargin=0.7em,%
	middlelinewidth=.2em,%
	linecolor=black,%
	fontcolor=black,%
	firstextra={\path let \p1=(P), \p2=(O) in ($(\x2,0)+0.5*(0,\y1)$)
										node[warningsymbol] {$\mathcal{S}$};},%
	secondextra={\path let \p1=(P), \p2=(O) in ($(\x2,0)+0.5*(0,\y1)$)
										node[warningsymbol] {$\mathcal{S}$};},%
	middleextra={\path let \p1=(P), \p2=(O) in ($(\x2,0)+0.5*(0,\y1)$)
										node[warningsymbol] {$\mathcal{S}$};},%
	singleextra={\path let \p1=(P), \p2=(O) in ($(\x2,0)+0.5*(0,\y1)$)
										node[warningsymbol] {$\mathcal{S}$};},%
}
%%%%%%%%%%%%%%%%%%%%%%%%%%%%%%%%%%% Tema - BEGIN
\newtheoremstyle{Tema}% name of the style to be used
  {0mm}% measure of space to leave above the theorem. E.g.: 3pt
  {10mm}% measure of space to leave below the theorem. E.g.: 3pt
  {}% name of font to use in the body of the theorem
  {}% measure of space to indent
  {\bfseries}% name of head font
  {\newline}% punctuation between head and body
  {30mm}% space after theorem head
  {}% Manually specify head

\theoremstyle{Tema} \newtheorem{Tema}{Tema} %%%%% Template para Temas
\theoremstyle{Tema} \newtheorem{serie}{Serie}              %%%%%  Template para Series de ejercicios
\theoremstyle{Tema} \newtheorem{teorema}{Teorema}              %%%%%  Template para Teoremas
\theoremstyle{Tema} \newtheorem{pregunta}{Pregunta}              %%%%%  Template para Series de ejercicios
\theoremstyle{Tema} \newtheorem{ejercicio}{Ejercicio}    %%%%%  Template para Ejercicios
\theoremstyle{Tema} \newtheorem{ejemplo}{Ejemplo}    %%%%%  Template para Ejemplos
\theoremstyle{Tema} \newtheorem{solucion}{Solución}    %%%%%  Template para Soluciones
\theoremstyle{Tema} \newtheorem{problem}{Problema}    %%%%%  Template para Problema
\theoremstyle{Tema} \newtheorem{definicion}{Definición}    %%%%%  Template para Soluciones
\theoremstyle{Tema} \newtheorem{proposicion}{Proposición}    %%%%%  Template para Soluciones
\theoremstyle{Tema} \newtheorem{lema}{Lema}    %%%%%  Template para Soluciones
\theoremstyle{Tema} \newtheorem{reto}{Reto}    %%%%%  Template para Reto
%-------------------------END-------------------------------------


%%%%%%%%%%%%%%%%%%%%%%%%%%%%%%%%%%%%%%%%%%%%%%%%%%%%%%%%%%%
\usepackage{fancyhdr}%formato de pagina
\pagestyle{fancy}%colocar la pagina con el formato deseado
\fancyhead{}
%\fancyhead[L]{\footnotesize{Sistemas Dinámicos}}  
%\fancyhead[C]{Corto 1}
\fancyhead[R]{\footnotesize{\thepage}}
%\fancyhead[LO,RE]{Cálculo 3}
%\fancyhead[RO,LE]{\footnotesize{\thepage}}
\fancyfoot{}
%\fancyfoot[L]{Diego Sarceño}
%\fancyfoot[LO,RE]{Diego Sarceño}
%%%%%%%%%%%%%%%%%%%%%%%%%%%%%%%%%%%%%%%%%%%%%%%%%%%%%%%%%%%
%% NUEVA BARRA INFERIOR, NICEEEE :3
\usepackage{fourier-orns}

\renewcommand\footrule{%
\hrulefill
\raisebox{-2.1pt}
{\quad\decosix\quad}%
\hrulefill}
%%%%%%%%%%%%%%%%%%%%%%%%%%%%%%%%%%%%%%%%%%%%%%%%%%%%%%%%%%%
\newcommand{\inner}[2]{\langle #1 , #2 \rangle}
\newcommand{\metric}[2]{\rho(#1,#2)}	
\newcommand{\seque}[2]{\{ #1_{#2} \}}
%%%%%%%%%%%%%%%%%%%%%%%%%%%%%%%%%%%%%%%%%%%%%%%%%%%%%%%%%%%
\definecolor{DS_Black}{HTML}{000000}

\begin{document}

%% AUTOR: Diego Sarceño

% ENCABEZADO DE TRABAJOS CON LOGO DE LA UNIDAD ACADÉMICA

% ENCABEZADO LOGO COLOR
%\begin{tabulary}{20cm}{Lp{0.9cm}p{6.1cm}}
%Universidad de San Carlos de Guatemala & & \multirow{4}{8cm}{\hfill %\includegraphics[scale=0.5]{ECFM.png}}\\            % Logo de la unidad academica
%Escuela de Ciencias Físicas y Matemáticas & \hfill & \\
%Diego Sarceño 201900109 & \hfill & \\
%Análisis de Variable Compleja 1 & \hfill & \\
%\today & & \\
%\end{tabulary}\\[0.25cm]


% ENCABEZADO LOGOS
%\noindent 
%\begin{tabulary}{20cm}{LLCRRR}
%\multirow{4}{2.3cm}{\includegraphics[scale=0.13]{/home/diego/Documents/Licenciatura/LatexBasic/ECFM.pdf}} & Universidad de San Carlos de Guatemala  &  & ~\hfill & & \multirow{4}{4.3cm}{\hfill \includegraphics[scale=0.082]{/home/diego/Documents/Licenciatura/LatexBasic/USAC.pdf}}\tabularnewline
% & Escuela de Ciencias Físicas y Matemáticas & ~\hfill &  & & \tabularnewline
% & Sistemas Dinámicos, Semestre 2, 2023 & & &  & \tabularnewline
% & Profesor: Jose Carlos Bonilla & & & & \tabularnewline
% & Alumno: Diego Sarceño, 201900109 & & & & \tabularnewline
%\end{tabulary}\\[0.75cm]
%
%%{\hrule height 1.5pt} \vspace{0.1cm}
%%\begin{tabulary}{21cm}{p{5.5cm}p{11.5cm}p{2cm}}
%%    \hfill & \huge{\scshape{Guía 3}} & \hfill
%%\end{tabulary}
%%{\hrule height 1.5pt} 
%%\vspace{0.5cm}
%
%
%{\hrule height 1.5pt}
%\begin{center}
%	\huge{\scshape{Corto 1}}
%\end{center}
%{\hrule height 1.5pt} 






\textcolor{DS_Black}{
\begin{minipage}{0.85\textwidth}
    \begin{center}
        \textbf{\Large Tarea 1}\\
        \vspace{5pt}
        Física Atmosférica \\
        \vspace{20pt}
        \textit{Diego Sarceño} \\
        \vspace{5pt}
        \footnotesize{\textit{201900109}} \\
        \vspace{5pt}
        \today
    \end{center}
\end{minipage}
\vspace{10pt}
\hrule
}
%
%\textcolor{DS_Black}{
%\begin{minipage}{0.85\textwidth}
%    \begin{center}
%        \textbf{\Large Estudio Estadístco en Competiciones Deportivas}\\
%        \vspace{5pt}
%        Propuesta Proyecto de Prácticas \\
%        \vspace{20pt}
%        \textit{Diego Sarceño} \\
%        \vspace{5pt}
%        \footnotesize{\textit{201900109}} \\
%        \vspace{5pt}
%        \today
%    \end{center}
%\end{minipage}
%\vspace{10pt}
%\hrule
%}


\begin{titlepage}


\begin{flushleft}
    Universidad de San Carlos de Guatemala \\
    Escuela de Ciencias Físicas y Matemáticas \\
    Prácticas Finales \\
    Asesor: MSc. Damián Ochoa
\end{flushleft}

\vspace{7.5cm}

\begin{center}
    \huge{\textsc{Estadístico en Competiciones Deportivas}} \\[1cm]
\end{center}

\vspace{7.5cm}

\begin{flushright}
    Diego Sarceño \\
    $201900109$
\end{flushright}

\vspace{1cm}

\begin{center}
    Guatemala, 31 de enero del 2024
\end{center}

\end{titlepage}


%\noindent \textbf{Instrucciones: } Resuelva cada uno de los siguientes problemas a \LaTeX  o a mano con letra clara y legible, dejando constancia de sus procedimientos. No es necesaria la carátula, únicamente su identificaciónn y las respuestas encerradas en un cuadro.

%\vspace{1cm}

\section{Introducción}

\section{Objetivos}
\subsection{General}
\subsection{Específicos}



\section{Justificación}



\section{Marco Teórico}



\section{Metodología}




\section{Cronograma}















%%%%


\section{Introducción}





\section{Objetivos}
\subsection{General}
\begin{itemize}
	\item Estudiar la estadística y los procesos detrás de los programas de predicción probabilística de resultados deportivos.
\end{itemize}

\subsection{Específicos}
\begin{enumerate}
	\item Recolectar artículos con la teoría relacionada.
	\item Recolección de datos necesarios para simulación.
	\item Realización de simulaciones con los datos obtenidos en base a la teoría revisada.
\end{enumerate}


\section{Justificación}


%\section{Marco Teórico}



\section{Metodología}






\section{Cronograma}




%\begin{thebibliography}{00}
%\bibitem{b1} Falomir, H. (2015). \textit{Curso de métodos de la física matemática.} Series: Libros de Cátedra.
%\bibitem{b2} Saxe, K. (2002). \textit{Beginning functional analysis (p. 7)}. New York: Springer.
%\bibitem{b3} Reed, M. (2012). \textit{Methods of modern mathematical physics: Functional analysis.} Elsevier.
%\bibitem{b4} Axler, S. (2015). \textit{Linear algebra done right.} springer publication.
%\bibitem{b5} Bartle, R. G., \& Sherbert, D. R. (2000). \textit{Introduction to real analysis.} John Wiley \& Sons, Inc..
%\end{thebibliography}




\end{document}

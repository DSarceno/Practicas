 \input{/home/diego/Documents/Licenciatura/LatexBasic/Preamble_general}

%%%%%%%%%%%%%%%%%%%%%%%%%%%%%%%%%%%%%%%%%%%%%%%%%%%%%%%%%%%
\usepackage{fancyhdr}%formato de pagina
\pagestyle{fancy}%colocar la pagina con el formato deseado
\fancyhead{}
%\fancyhead[L]{\footnotesize{Sistemas Dinámicos}}  
%\fancyhead[C]{Corto 1}
\fancyhead[R]{\footnotesize{\thepage}}
%\fancyhead[LO,RE]{Cálculo 3}
%\fancyhead[RO,LE]{\footnotesize{\thepage}}
\fancyfoot{}
%\fancyfoot[L]{Diego Sarceño}
%\fancyfoot[LO,RE]{Diego Sarceño}
%%%%%%%%%%%%%%%%%%%%%%%%%%%%%%%%%%%%%%%%%%%%%%%%%%%%%%%%%%%
%% NUEVA BARRA INFERIOR, NICEEEE :3
\usepackage{fourier-orns}

\renewcommand\footrule{%
\hrulefill
\raisebox{-2.1pt}
{\quad\decosix\quad}%
\hrulefill}
%%%%%%%%%%%%%%%%%%%%%%%%%%%%%%%%%%%%%%%%%%%%%%%%%%%%%%%%%%%
\newcommand{\inner}[2]{\langle #1 , #2 \rangle}
\newcommand{\metric}[2]{\rho(#1,#2)}	
\newcommand{\seque}[2]{\{ #1_{#2} \}}
%%%%%%%%%%%%%%%%%%%%%%%%%%%%%%%%%%%%%%%%%%%%%%%%%%%%%%%%%%%
\definecolor{DS_Black}{HTML}{000000}

\begin{document}

%\input{Header_original}
%
%\textcolor{DS_Black}{
%\begin{minipage}{0.85\textwidth}
%    \begin{center}
%        \textbf{\Large Estudio Estadístco en Competiciones Deportivas}\\
%        \vspace{5pt}
%        Propuesta Proyecto de Prácticas \\
%        \vspace{20pt}
%        \textit{Diego Sarceño} \\
%        \vspace{5pt}
%        \footnotesize{\textit{201900109}} \\
%        \vspace{5pt}
%        \today
%    \end{center}
%\end{minipage}
%\vspace{10pt}
%\hrule
%}


\begin{titlepage}


\begin{flushleft}
    Universidad de San Carlos de Guatemala \\
    Escuela de Ciencias Físicas y Matemáticas \\
    Prácticas Finales \\
    Asesor: MSc. Damián Ochoa
\end{flushleft}

\vspace{7.5cm}

\begin{center}
    \huge{\textsc{Estadístico en Competiciones Deportivas}} \\[1cm]
\end{center}

\vspace{7.5cm}

\begin{flushright}
    Diego Sarceño \\
    $201900109$
\end{flushright}

\vspace{1cm}

\begin{center}
    Guatemala, 31 de enero del 2024
\end{center}

\end{titlepage}


%\noindent \textbf{Instrucciones: } Resuelva cada uno de los siguientes problemas a \LaTeX  o a mano con letra clara y legible, dejando constancia de sus procedimientos. No es necesaria la carátula, únicamente su identificaciónn y las respuestas encerradas en un cuadro.

%\vspace{1cm}

\section{Introducción}

\section{Objetivos}
\subsection{General}
\subsection{Específicos}



\section{Justificación}



\section{Marco Teórico}



\section{Metodología}




\section{Cronograma}















%%%%


\section{Introducción}





\section{Objetivos}
\subsection{General}
\begin{itemize}
	\item Estudiar la estadística y los procesos detrás de los programas de predicción probabilística de resultados deportivos.
\end{itemize}

\subsection{Específicos}
\begin{enumerate}
	\item Recolectar artículos con la teoría relacionada.
	\item Recolección de datos necesarios para simulación.
	\item Realización de simulaciones con los datos obtenidos en base a la teoría revisada.
\end{enumerate}


\section{Justificación}


%\section{Marco Teórico}



\section{Metodología}






\section{Cronograma}




%\begin{thebibliography}{00}
%\bibitem{b1} Falomir, H. (2015). \textit{Curso de métodos de la física matemática.} Series: Libros de Cátedra.
%\bibitem{b2} Saxe, K. (2002). \textit{Beginning functional analysis (p. 7)}. New York: Springer.
%\bibitem{b3} Reed, M. (2012). \textit{Methods of modern mathematical physics: Functional analysis.} Elsevier.
%\bibitem{b4} Axler, S. (2015). \textit{Linear algebra done right.} springer publication.
%\bibitem{b5} Bartle, R. G., \& Sherbert, D. R. (2000). \textit{Introduction to real analysis.} John Wiley \& Sons, Inc..
%\end{thebibliography}




\end{document}
